\section*{Discussion}
\jri{this paragraph needs reworking and to jive with the below from peter}
The rarity of convergent evolution in maize contrasts with data from humans \citep{Tennessen_2011_21698142} showing selection on the same genes in multiple pairs of tropical and temperate populations.  
However, in both maize and humans the majority of adaptive variants appear to have been derived from standing variation \cite[]{Tennessen_2011_21698142}.
One difference between these two species is effective population size: the effective population size of maize ($\sim10^5$) is an order of magnitude larger than that in humans \cite[]{Takahata_1997_9114074}.
Humans would therefore have less standing variation as a source of adaptation, perhaps resulting in the same variants being selected in multiple subpopulations (as long as $s$ is sufficiently large and initial frequency is, for example, $>0.1$).
In contrast, maize could maintain a larger number of adaptive variants in the ancestral lowland population.
In this case, if genetic variants produce similar phenotypic effects, they may be selected to high frequency in independent highland regions at random.
Or if Mexican and S. American highlands have slightly different climates, it is feasible that different variants are selected.
The target size of mutations can also increase the variants for adaptation, but there are currently no data regarding mutational target size in maize versus humans.
\plr{You are discussing adaptation from standing variation; what about from new mutation?  This wouldn't require postulating lots of equivalent variants?} \mbh{Sho's data seem to suggest predominance of adaptation from standing variation.  But, given that maize populations are larger than human populations, new mutations would also be more common in maize, which would, I think, lead to less convergence in maize than humans}
In fact, adaptation from multiple standing variants that produce similar phenotypes has previously been observed in maize: the \emph{grassy tillers1} (\emph{gt1}) gene \cite[]{Wills_2013_23825971} contains two artificially selected mutations that reduce ear number.
These mutations both segregate at low frequency in \emph{parviglumis} but have been individually selected to high frequency in different populations of maize.

\section*{Conclusions} \jri{ WE NEED A CONCLUSION! }
We successfully inferred/revealed the molecular basis of independent adaptation to highland climates in the Mexican and S. American maize populations.
For this aim, we utilized the high throughput genotyping technologies: GBS and Illumina MaizeSNP50 BeadChip platform.
We inferred the demography, and detected the candidates of adaptive loci to highland climates by analyzing $\approx 90,000$ SNPs.
The majority of adaptive variants were derived from standing genetic variation in the maize with a large effective population size as theory predicted.
Introgression events from wild highland maize are also important source of highland adaptation.
Surprisingly, despite the fact that the environments of Mexican and S. American highlands are very similar, we found that parallel adaptation is very rare. 
Our newly developed theory supported the rarity of parallel adaptation in maize (???).
Thus, maize highland adaptation could be a good example of convergent adaptation in natural populations.

%\st{silly sentences.}
%However, it is known that the effective recombination rate of maize is very high (CITE, Maud, PNAS).
%Linkage disequilibrium rapidly decays and reaches plateau around 100 bp (Figure~\ref{supp:LD}).
%The density of SNPs was roughly 1 SNP per 20 kb, so we could miss some of the adaptive variants.
%We believe that our conclusion holds even if more dense SNP dataset can be used, but the future genotyping technology may solve this problem.

%1. We successfully inferred demography and detected the candidates of adaptive loci to highland climates in Mexico and South America by utilizing GBS and 55-k chip.

%2. The main conclusion is parallel adaptation is rare in maize highland adaptation.
