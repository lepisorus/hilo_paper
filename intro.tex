\section*{Introduction}
\noindent Convergent evolution occurs when multiple species or populations exhibit similar phenotypic adaptations to comparable environmental challenges \cite[]{Wood_2005_15881688,Arendt_2008_18022278,Elmer_2011_21459472}.
Evolutionary genetic analysis of a wide range of species has provided evidence for multiple pathways of convergent evolution. 
One such route occurs when identical mutations arise independently and fix via natural selection in multiple populations. 
In humans, for example, malaria resistance due to mutations from Glu to Val at the sixth codon of the $\beta$-globin gene has arisen independently on multiple unique haplotypes  \cite[]{Currat_2002_11741197,Kwiatkowski_2005_16001361}.  
Convergent evolution can also be achieved when different mutations arise within the same locus yet produce similar phenotypic effects.  
Grain fragrance in rice appears to have evolved along these lines, as populations across East Asia have similar fragrances resulting from at least eight distinct loss-of-function alleles in the  \emph{BADH2} gene \cite[]{Kovach_2009_19706531}.  
Finally, convergent evolution may arise from natural selection acting on standing genetic variation in an ancestral population.  
In the three-spined stickleback, natural selection has repeatedly acted to reduce armor plating in independent colonizations of freshwater environments.  
Adaptation in these populations occurred both from new mutations as well as standing variation at the \emph{Eda} locus in marine populations \cite[]{Colosimo_2005_15790847}.  

%The same gene have contributed to multiple adaptation events in these examples that have been classically referred to as ``parallel adaptation'' \cite[]{Wood_2005_15881688}.
%\jri{another recent nice example, repeated gene loss in cyanogenic clovers \url{http://rstb.royalsocietypublishing.org/content/369/1648/20130347} or convergene evolution of electric shocks \url{http://www.sciencemag.org/content/344/6191/1522}; also feel like we need more intro here.  this mentions a few ideas, but not some of the interesting questions or relevant points about genetic architecture, target size, the role of migration, etc.}

Not all convergent phenotypic evolution is the result of convergent evolution at the molecular level, however.  
Recent studies of adaptation to high elevation in humans, for example, reveal that the genes involved in highland adaptation are largely distinct among Tibetan, Andean and Ethiopian populations \cite[]{Bigham_2010_20838600,Scheinfeldt_2012_22264333,Alkorta-Aranburu_2012_23236293}. 
While observations of independent origin may be due to a complex genetic architecture or standing genetic variation, introgression from related populations may also play a role.  
In Tibetan populations, the adaptive allele at the \emph{EPAS1} locus appears to have arisen via introgression from Denisovans, a related hominid group \cite[]{huerta2014altitude}.
Overall, we still know relatively little about how convergent phenotypic evolution is driven by common genetic changes or the relative frequencies of these different routes of convergent evolution.

The adaptation of maize to high elevation environments (\emph{Zea mays} ssp. \emph{mays}) provides an excellent opportunity to investigate the molecular basis of convergent evolution.  
Maize was domesticated from the wild teosinte \emph{Zea mays} ssp. \emph{parviglumis} (hereafter \emph{parviglumis}) in the lowlands of southwest Mexico $\sim$9,000 years before present (BP) \cite[]{Matsuoka_2002_11983901,Piperno_2009_19307570,vanHeerwaarden_2011_21189301}. 
After domestication, maize spread rapidly across the Americas, reaching the lowlands of South America and the high elevations of the Mexican Central Plateau by $\sim 6,000$ BP \cite[]{Piperno_2006_69}, and the Andean highlands by $\sim 4,000$ 
BP \cite[]{Perry_2006_16511492,Grobman_2012_22307642}. 
The transition from lowland to highland habitats spanned similar environmental gradients in Mexico and South America (Figure~\ref{supp:colfreq}) and presented a host of novel challenges that often accompany highland adaptation including reduced temperature, increased ultraviolet radiation, and reduced partial pressure of atmospheric gases \cite[]{Korner_2007_17988759}.

Common garden experiments in Mexico reveal that highland maize has successfully adapted to high elevation conditions \cite[]{Mercer2008}, and phenotypic comparisons between Mexican and South American populations are suggestive of convergent evolution.  
Maize landraces (open-pollinated traditional varieties) from both populations share a number of phenotypes not found in lowland populations, including dense macrohairs \cite[]{Wilkes_1977,Wellhausen1957:book}, stem pigmentation \cite[]{Wilkes_1977,Wellhausen1957:book}, differences in tassel branch and ear husk number \cite[]{brewbaker2014diversity}, and biochemical response to UV radiation \cite[]{Casati2005}. 
In spite of these shared phenotypes, genetic analyses of maize landraces from across the Americas indicate that the two highland populations are independently derived from their respective lowland populations \cite[]{Vigouroux_2008_21632329, vanHeerwaarden_2011_21189301}, suggesting that observed patterns of phenotypic similarity are not simply due to recent shared ancestry. 

In addition to convergent evolution between maize landraces, a number of lines of evidence suggest convergent evolution in the related wild teosintes.  
\emph{Zea mays} ssp. \emph{mexicana} (hereafter \emph{mexicana}) is native to the highlands of central Mexico, where it is thought to have occurred since at least the last glacial maximum \cite[]{Ross-Ibarra_2009_19153259, Hufford_niche}. 
Phenotypic differences between \emph{mexicana} and the lowland \emph{parviglumis} mirror those between highland and lowland maize \cite[]{Lauter_2004_15342532}, and population genetic analyses of the two subspecies reveal evidence of natural selection associated with altitudinal differences between \emph{mexicana} and \emph{parviglumis} \cite[]{Pyhajarvi2013,fang2012megabase}.  
Landraces in the highlands of Mexico are often found in sympatry with \emph{mexicana} and gene flow from \emph{mexicana} likely contributed to maize adaptation to the highlands \cite[]{Profford_2013}. 
No wild \emph{Zea} occur in S. America, and S. American landraces show no evidence of gene flow from Mexican teosinte \cite[]{vanHeerwaarden_2011_21189301}, further suggesting an independent origin of convergent phenotypic adaptation.

Here we use genome-wide SNP data from Mexican and S. American landraces to investigate the evidence for convergent evolution to highland environments at the molecular level.  
We estimate demographic histories for maize in the highlands of Mexico and South America, then use these models to identify loci that may have been the target of selection in each population.
We find a large number of sites showing evidence of selection, consistent with a complex genetic architecture involving many phenotypes and numerous loci.  
We see little evidence for shared selection at the nucleotide or gene level, a result we show is consistent with expectations from recent theoretical work on convergent adaptation \cite[]{ralph2014convergent}.
Instead, our results support a role of adaptive introgression from teosinte in Mexico and highlight the contribution of standing variation to adaptation in both populations.
