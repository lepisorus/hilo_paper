\section*{Introduction}
%
% are there more examples or more routes we are missing and should mention?
%
%
%do we cite lecorre and kremer? we don't really get at multilocus selection on quant. traits though. if not them, who do we cite here?
%


The transition from lowland to highland habitats spanned similar environmental gradients in Mexico and South America (supp fig. \ref{bioclim}) and presented a host of novel challenges including  temperature, precipitation, ultraviolet radiation, and hypoxia. 
%
% do we include  hypoxia? i don't know and haven't looked what lit is available on hypoxia (is that even the right term?)
%
% for UV we need citations to relevant literature: e.g. http://www.biomedcentral.com/1471-2229/12/92 and http://www.reeis.usda.gov/web/crisprojectpages/0195696-maize-responses-to-uv-b-a-genomics-assessment.html for a list of some relevant pubs.
%
% Shohei please check to see if we get hits for genes found important for UV radiation!  see: 
%
% this paper even shows that highland maize has a novel flavonoid regulation in response to UV not seen in lowland lines, and that it's present in both Mexican and Andes! http://onlinelibrary.wiley.com/store/10.1111/j.1365-3040.2005.01329.x/asset/j.1365-3040.2005.01329.x.pdf?v=1&t=hj58i8lj&s=c44c9bf9d093c22371bd10478924b041a1c3ef9b
%
% Matt please add citations to the above sentence on novel conditions if you think they're needed
%
Common garden experiments in Mexico reveal that highland maize has successfully adapted to highland conditions \cite[]{Mercer2008}, and phenotypic comparisons between Mexican and South American populations are suggestive of parallel adaptation.  Landraces from both populations share a number of phenotypes not found in lowland populations, including dense macrohairs \cite[]{CITE}, stem pigmentation \cite[]{CITE}, biochemical response to UV radiation \cite[]{Casati2005}, and...
%
% Matt please fill in if there are other features you're aware of. 
%
%I think some of these cites need to be included above?
%It has been reported that \emph{mexicana} and highland Mexican maize share morphological features including traits presumably involved in highland adaptation.  \cite[]{Collins_1921,Wilkes1967:book,Wilkes_1977,Lauter_2004_15342532}

Although there are no wild relatives of maize in South America, the teosinte \emph{Zea mays} ssp. \emph{mexicana} (hereafter \emph{mexicana}) is native to the highlands of central Mexico, where it is thought to have occurred at least since the last glacial maximum \cite[]{Ross-Ibarra 2009, Hufford_niche}. Phenotypic differences between \emph{mexicana} and \emph{parviglumis} mirror those between highland and lowland maize \cite[]{Lauter_2004_15342532} and population genetic analyses of the two subspecies reveal evidence of natural selection associated with altitude \emph{mexicana} and \emph{parviglumis} \cite[]{Pyhajarvi2013}.  Landraces in the highlands of Mexico are often found in sympatry with  \emph{mexicana}, and gene flow between the two is thought to have contributed to maize adaptation \cite[]{Profford_2013}.

In this paper we set out to address a number of questions regarding highland adaptation in maize: What is the genetic architecture of highland adaptation? Do maize populations in South America show evidence of parallel adaptation? 
We make use of SNP genotyping to characterize patterns of natural selection in highland maize from both Mexico and South America, and compare our results to expectations from theoretical models of parallel adaptation.  We find X, Y, Z.
 %I welcome suggestions/edits on questions (I think we should add a few more?) and someone take a stab at the XYZ part describing what we find.
