\section*{Introduction}
\noindent Parallel adaptation is defined as a process in which multiple species or subpopulations independently adapt to similar environments via mutations at the same loci \cite[]{Wood_2005_15881688,Arendt_2008_18022278,Elmer_2011_21459472}.
%is this a definition we are all happy with? ST
% ST : agreed, and added some references
Evolutionary genetic analysis of a wide range of species have provided evidence for multiple pathways to parallel adaptation. One route to parallel adaptation occurs when independent identical beneficial mutations arise and fix via natural selection in multiple populations. In humans, for example, malaria resistance due to mutations from Glu to Val at the sixth codon of the $\beta$-globin gene have arisen independently on multiple unique haplotypes  \cite[]{Currat_2002_11741197,Kwiatkowski_2005_16001361}.  We also consider as parallel adaptation cases in which different mutations arise within the same locus and produce similar phenotypic effects.  Grain fragrance in rice appears to have evolved along these lines, as populations across East Asia have similar fragrances resulting from at least eight distinct loss-of-function alleles in the  \emph{BADH2} gene \cite[]{Kovach_2009_19706531}.  Finally, parallel adaptation may arise from natural selection acting on standing genetic variation in an ancestral population.  In the threespine stickleback natural selection has repeatedly acted to reduce armor plating in multiple independent colonizations of freshwater environments.  In most cases, adaptation in these populations took advantage of standing variation at the \emph{Eda} locus in marine populations \cite[]{Colosimo_2005_15790847}.  
%
% are there more examples or more routes we are missing and should mention?
%

%
% ST: I think we include all routes (but keep checking!!) and added some reviews in the first para, so, in my opinion, the examples is enough
%
%While these examples of single-locus selection are relatively simple, in reality an organism must adapt to a number of abiotic and biotic conditions and the genetic architecture of adaptive phenotypes may be complex.  
%
%do we cite lecorre and kremer? we don't really get at multilocus selection on quant. traits though. if not them, who do we cite here?
%
%A number of questions regarding the nature of parallel adaptation remain to be studied at this scale, including how often parallelism at the phenotypic level is reflected by parallel adaptation at the molecular level, and whether parallel adaptation occurs primarily via new mutations or from standing variation.   

%
% ST : we may be able to write more simply like below (just an idea). 
%
A lot of examples of parallel adaptation within species lead us to ask some questions such as 
i) how often parallelism at the phenotypic level is reflected by parallel adaptation at the molecular level, and 
ii) whether parallel adaptation occurs primarily via new mutations or from standing variation.
To do this, a genome scan method would be feasible to identify adaptive loci that are involved in multiple subpopulations.

% from the discussion of a previous version.  Not sure this sentence should be here or later
\cite{Tennessen_2011_21698142} more frequently observed the adaptation in the same genes than neutral expectation when comparing the divergence between multiple pairs of tropical and temperate human populations.  This is a nice example of a genome scan. 

Domesticated maize (Zea mays ssp. mays) provides us an great opportunity to investigate the molecular basis of parallel adaptation to highland conditions.  Maize was domesticated from the wild teosinte \emph{Zea mays} ssp. \emph{parviglumis} (hereafter \emph{parviglumis}) in the lowlands of southwest Mexico $\sim$9,000 years before present (BP) \cite[]{Matsuoka_2002_11983901,Piperno_2009_19307570,vanHeerwaarden_2011_21189301}. After domestication, maize spread rapidly across the Americas, reaching the lowlands of South America and the high altitudes of the Mexican central plateau by $\sim$6,000 BP \cite[]{Piperno_2006_69}, and the Andean highlands $\sim$2,000 years later \cite[]{Perry_2006_16511492,Grobman_2012_22307642}. Genetic analyses of maize landraces from across the Americas strongly suggest that the two highland populations are independently derived from their respective lowland populations \cite[]{Vigouroux_2008_21632329, vanHeerwaarden_2011_21189301}. 

The transition from lowland to highland habitats spanned similar environmental gradients in Mexico and South America (supp fig. \ref{bioclim}) and presented a host of novel challenges including  temperature, precipitation, ultraviolet radiation, and hypoxia. 
%
% do we include  hypoxia? i don't know and haven't looked what lit is available on hypoxia (is that even the right term?)
%
% for UV we need citations to relevant literature: e.g. http://www.biomedcentral.com/1471-2229/12/92 and http://www.reeis.usda.gov/web/crisprojectpages/0195696-maize-responses-to-uv-b-a-genomics-assessment.html for a list of some relevant pubs.
%
% Shohei please check to see if we get hits for genes found important for UV radiation!  see: 
%
% this paper even shows that highland maize has a novel flavonoid regulation in response to UV not seen in lowland lines, and that it's present in both Mexican and Andes! http://onlinelibrary.wiley.com/store/10.1111/j.1365-3040.2005.01329.x/asset/j.1365-3040.2005.01329.x.pdf?v=1&t=hj58i8lj&s=c44c9bf9d093c22371bd10478924b041a1c3ef9b
%
% Matt please add citations to the above sentence on novel conditions if you think they're needed
%
Common garden experiments in Mexico reveal that highland maize has successfully adapted to highland conditions \cite[]{Mercer2008}, and phenotypic comparisons between Mexican and South American populations are suggestive of parallel adaptation.  Landraces from both populations share a number of phenotypes not found in lowland populations, including dense macrohairs \cite[]{CITE}, stem pigmentation \cite[]{CITE}, biochemical response to UV radiation \cite[]{Casati2005}, and...
%
% Matt please fill in if there are other features you're aware of. 
%
%I think some of these cites need to be included above?
%It has been reported that \emph{mexicana} and highland Mexican maize share morphological features including traits presumably involved in highland adaptation.  \cite[]{Collins_1921,Wilkes1967:book,Wilkes_1977,Lauter_2004_15342532}

Although there are no wild relatives of maize in South America, the teosinte \emph{Zea mays} ssp. \emph{mexicana} (hereafter \emph{mexicana}) is native to the highlands of central Mexico, where it is thought to have occurred at least since the last glacial maximum \cite[]{Ross-Ibarra 2009, Hufford_niche}. Phenotypic differences between \emph{mexicana} and \emph{parviglumis} mirror those between highland and lowland maize \cite[]{Lauter_2004_15342532} and population genetic analyses of the two subspecies reveal evidence of natural selection associated with altitude \emph{mexicana} and \emph{parviglumis} \cite[]{Pyhajarvi2013}.  Landraces in the highlands of Mexico are often found in sympatry with  \emph{mexicana}, and gene flow between the two is thought to have contributed to maize adaptation \cite[]{Profford_2013}.

In this paper we set out to address a number of questions regarding highland adaptation in maize: What is the genetic architecture of highland adaptation? Do maize populations in South America show evidence of parallel adaptation? 
We make use of SNP genotyping to characterize patterns of natural selection in highland maize from both Mexico and South America, and compare our results to expectations from theoretical models of parallel adaptation.  We find X, Y, Z.
 %I welcome suggestions/edits on questions (I think we should add a few more?) and someone take a stab at the XYZ part describing what we find.
