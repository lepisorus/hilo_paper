\section*{Introduction}
\noindent Parallel adaptation is a process in which multiple species or populations independently adapt to distinct regions with similar environments via mutations at the same loci \cite[]{Wood_2005_15881688,Arendt_2008_18022278,Elmer_2011_21459472}.
Evolutionary genetic analysis of a wide range of species has provided evidence for multiple pathways to parallel adaptation. One such route occurs when identical mutations arise independently and fix via natural selection in multiple populations. In humans, for example, malaria resistance due to mutations from Glu to Val at the sixth codon of the $\beta$-globin gene has arisen independently on multiple unique haplotypes  \cite[]{Currat_2002_11741197,Kwiatkowski_2005_16001361}.  Parallel adaptation can also be achieved when different mutations arise within the same locus and produce similar phenotypic effects.  Grain fragrance in rice appears to have evolved along these lines, as populations across East Asia have similar fragrances resulting from at least eight distinct loss-of-function alleles in the  \emph{BADH2} gene \cite[]{Kovach_2009_19706531}.  Finally, parallel adaptation may arise from natural selection acting on standing genetic variation in an ancestral population.  In the three-spined stickleback, natural selection has repeatedly acted to reduce armor plating in independent colonizations of freshwater environments.  In most cases, adaptation in these populations took advantage of standing variation at the \emph{Eda} locus in marine populations \cite[]{Colosimo_2005_15790847}.  

We still know relatively little, however, about how common parallel phenotypic evolution is driven by parallel genetic changes or the relative frequencies of these different routes of parallel adaptation.
Domesticated maize (\emph{Zea mays} ssp. \emph{mays}) provides an excellent opportunity to investigate the molecular basis of parallel adaptation.  Maize was domesticated from the wild teosinte \emph{Zea mays} ssp. \emph{parviglumis} (hereafter \emph{parviglumis}) in the lowlands of southwest Mexico $\sim$9,000 years before present (BP) \cite[]{Matsuoka_2002_11983901,Piperno_2009_19307570,vanHeerwaarden_2011_21189301}. After domestication, maize spread rapidly across the Americas, reaching the lowlands of South America and the high altitudes of the Mexican Central Plateau by $\sim$6,000 BP \cite[]{Piperno_2006_69}, and the Andean highlands $\sim$2,000 years later \cite[]{Perry_2006_16511492,Grobman_2012_22307642}. 
The transition from lowland to highland habitats spanned similar environmental gradients in Mexico and South America (supp fig. \ref{bioclim}) and presented a host of novel challenges that commonly, if not universally, accompany highland adaptation including reduced temperature, increased ultraviolet radiation, and reduced partial pressure of atmospheric gases \cite[]{Korner_2007_179887}.
%citation is "K�rner, C. 2007. The use of 'altitude' in ecological research. Trends in Ecology and Evolution 22:569-574 
% ST: Done!!

% for UV we need citations to relevant literature: e.g. http://www.biomedcentral.com/1471-2229/12/92 and http://www.reeis.usda.gov/web/crisprojectpages/0195696-maize-responses-to-uv-b-a-genomics-assessment.html for a list of some relevant pubs.
% Shohei please check to see if we get hits for genes found important for UV radiation!  see: 
% this paper even shows that highland maize has a novel flavonoid regulation in response to UV not seen in lowland lines, and that it's present in both Mexican and Andes! http://onlinelibrary.wiley.com/store/10.1111/j.1365-3040.2005.01329.x/asset/j.1365-3040.2005.01329.x.pdf?v=1&t=hj58i8lj&s=c44c9bf9d093c22371bd10478924b041a1c3ef9b

Common garden experiments in Mexico reveal that highland maize has successfully adapted to highland conditions \cite[]{Mercer2008}, and phenotypic comparisons between Mexican and South American populations are suggestive of parallel adaptation.  Landraces from both populations share a number of phenotypes not found in lowland populations, including dense macrohairs \cite[]{CITE,Wellhausen1957:book}, stem pigmentation \cite[]{CITE,Wellhausen1957:book}, and biochemical response to UV radiation \cite[]{Casati2005}. Genetic analyses of maize landraces from across the Americas indicate that the two highland populations are independently derived from their respective lowland populations \cite[]{Vigouroux_2008_21632329, vanHeerwaarden_2011_21189301}, so observed patterns of phenotypic similarity are not simply due to recent shared ancestry. 
%MBH: Lauter wouldn't be appropriate because it only talks about parviglumis and mexicana.  Wilkes comments that highland landraces have traits found in mexicana so that would probably be fine to add for macrohairs and stem pigmentation; alternatively I'm pretty sure the Wellhausen book on Mexican landraces and "Races of Maize" books from South America discuss macrohairs and pigment.  Do you still have these Jeff?  If not, I can probably dig them up at the library here (I'd be really bummed if ISU didn't have them!)
%It has been reported that \emph{mexicana} and highland Mexican maize share morphological features including traits presumably involved in highland adaptation.  \cite[]{Collins_1921,Wilkes1967:book,Wilkes_1977,Lauter_2004_15342532}
%Sho please add in Races of Maize citation (Wellhausen) and Wilkes above. What's the Collins ref?

Although there are no wild relatives of maize in South America, the teosinte \emph{Zea mays} ssp. \emph{mexicana} (hereafter \emph{mexicana}) is native to the highlands of central Mexico, where it is thought to have occurred since at least the last glacial maximum \cite[]{Ross-Ibarra_2009_19153259, Hufford_niche}. Phenotypic differences between \emph{mexicana} and \emph{parviglumis} mirror those between highland and lowland maize \cite[]{Lauter_2004_15342532} and population genetic analyses of the two subspecies reveal evidence of natural selection associated with altitudinal differences between \emph{mexicana} and \emph{parviglumis} \cite[]{Pyhajarvi2013}.  Landraces in the highlands of Mexico are often found in sympatry with  \emph{mexicana}, and gene flow between the two is thought to have contributed to maize adaptation to the highlands \cite[]{Profford_2013}.

In this paper we set out to address a number of questions regarding highland adaptation in maize: What is the genetic architecture of highland adaptation? Do maize populations in South America show evidence of parallel adaptation when compared with highland maize from Mexico? How do observed patterns of parallel adaptaiton compare to theoretical expectations?
We make use of SNP genotyping to characterize patterns of natural selection in highland maize and compare our results to expectations from theoretical models of parallel adaptation.  We find X, Y, Z.
 %I welcome suggestions/edits on questions (I think we should add a few more?) and someone take a stab at the XYZ part describing what we find.
 %MBH: Sho, I'm happy to write the X,Y,Z, but wanted to give you a crack at it first.  Feels strange for me to summarize what I think are the main points of your hard work! :-)
