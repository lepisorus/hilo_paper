\documentclass[onecolumn,oneside,letterpaper]{article} 

\usepackage{response}
%\usepackage{geneticsT2}

\usepackage{times}
\usepackage{color}
\usepackage{hyperref}

% rayout %
\addtolength{\oddsidemargin}{-2.75cm}
\addtolength{\evensidemargin}{-0.75cm}

\addtolength{\textwidth}{5.5cm}
\addtolength{\topmargin}{-3cm}
\addtolength{\textheight}{4.5cm}

%\parindent=4em
%\setlength{\parskip}{1ex plus 0.5ex minus 0.2ex} 

\setlength{\parindent}{0pt}
\setlength{\parskip}{10.0pt}
%\renewcommand{\itemindent}{50pt}
\setlength{\leftskip}{0.5cm}
\renewcommand{\labelitemi}{$-$}


\newcommand{\jri}[1]{\textcolor{blue}{ \emph{\scriptsize  #1}} }
\newcommand{\jristop}{\textcolor{red}{\hline \bf JRI STOPPED HERE\\ \hline}}
\newcommand{\st}[1]{\textcolor{red}{ #1}}
\newcommand{\comst}[1]{\textcolor{red}{ \em{\scriptsize  (#1)}} }
\definecolor{mattgreen} {rgb} {0,0.6,0}
\newcommand{\mbh}[1]{\textcolor{mattgreen}{ \em{\scriptsize  #1}} }
\definecolor{peterpurple}{rgb}{.6,0,.6}
\newcommand{\plr}[1]{\textcolor{peterpurple}{ \emph{\scriptsize (#1)}} }

\renewcommand{\textfraction}{0.01}
\renewcommand{\topfraction}{0.99}
\renewcommand{\bottomfraction}{0.65}
\renewcommand{\floatpagefraction}{0.90}
\renewcommand{\dbltopfraction}{0.95}
\renewcommand{\dblfloatpagefraction}{0.80}
\renewcommand{\sfdefault}{phv}

\usepackage{fancyhdr}
\pagestyle{fancy}
\fancyhf{}
\fancyfoot[CE,CO]{\thepage}
\renewcommand{\headrulewidth}{0pt}
\fancypagestyle{plain}{
	\fancyhf{}
}

\usepackage{natbib}
\bibpunct{(}{)}{;}{a}{}{,}

% space of double hline in Table
\doublerulesep = 0.4pt

\title{Parallel adaptation to highland climate in maize domesticated populations}

\author{
 Shohei Takuno, Kelly Swarts, Matthew B. Hufford, Rob J. Elshire, Jeffrey C. Glaubitz, Edward S. Buckler and Jeffrey Ross-Ibarra
   }

\usepackage{natbib}
\bibpunct{(}{)}{;}{a}{}{,}

\usepackage{amsmath}

\usepackage{graphicx}

\begin{document}

\maketitle
Dear Editor - 

Thank you for soliciting these reviews, which were very helpful.   We believe we have responded to all of the reviewers' comments, hopefully to their satisfaction.  We also hope that the manuscript is improved as a result of the revisions.  

We provide detailed responses to the reviewers' comments below; the responses are in \textit{italics} and also bulleted for ease of identification and edits to the text are highlighted in red.
Briefly, the revisions are as follows:  
We have moved the diffusion modeling to the appendix, as (as the AE pointed out)
this was made semi-redundant by the coalescent theory arguments.
We also added results regarding metabolic pathway analysis as reviewer 2 suggested.  
Finally, we fully incorporated the other minor comments into our revised version.  

I hope that these revisions will make the paper suitable for publication in Genetics. 

Thank you again! 

Sincerely, 

Jeffrey Ross-Ibarra

--------------------------------------------------------------------


\textbf{AE comments:}

The ``Demographic model" is too detailed.  These fine-scaled demographic factors influence just two parameters: 
the effective population density and the rate of local diffusion.  
Given all the uncertainties, it may be best just to consider a wide range of more or less plausible values. 

\response{
    We certainly agree that the demography should just come down to these parameters. We decided to explicitly work through the full model to avoid making mistakes in the process of handwaving
    (we find it hard to intuit the proper range of plausible offspring variances for a model like this!),
    and to be able to check approximations like that we use for the probability of fixation
    (as in Figure A1 in the appendix, which is done with the exact branching process).
    We've tried to make this more clear
    (see first paragraph in ``Theoretical evaluation'', around ``Much of the population genetics theory we use relies on universality results \ldots''),
    and tried in any case to discuss the effects of uncertainty in these critical parameters (in Comparison to Theory).
}

It is interesting to consider the relative plausibility that alleles that are favoured at high altitude either arose by mutation at high altitude, 
or moved from low altitudes where they are disfavoured.  
However, the actual model is not spelled out.  
Are these habitats disjunct?  
Or is there a continuous range? 
If the latter, does the selection coefficient change abruptly, or gradually?  
It seems that there is a gap of distance R between the two habitats, but then, how does the maize get across?  
This all seems obscure. 

\response{
    Apologies, we were working with a continuous model,
    approximating continuously-varying altitude by discrete patches of high/low elevation,
    and checked this approximation in the Appendix (around Figure A1).
    To make this more clear, we've separated out description of the geographic model into a paragraph ``Geographic distribution''.
}


More seriously, simple diffusion is not plausible for very long-term movement: 
assuming diffusion alone makes the chances of movement extremely sensitive to the parameters, 
as is apparent from the examples given.  
Ultimately, what matters is the number of new mutations in the highland population, 
relative to the numbers of mutations that arise in the lowland, 
persist despite counter-selection, and move, one way or another, to the highlands.  
It would be simpler just to say that rather than make over-elaborate diffusion models. 

\response{
    Since the coalescent theory calculations come to the same conclusion, and are more general,
    we've decided that the section involving migration-selection balance isn't essential to the paper,
    so we moved it to the Appendix.
    (We do think it's useful to include, as a further check and a comparison to the case of deleterious alleles, which we think is more realistic, as discussed in the paper.)
    We've also rewritten the first paragraph to ``Theoretical evaluation'',
    clarifying that we are interested in the three sources of convergence: new mutation, migration between highlands, and convergent selection on standing variation.
    Also note the section of the Discussion about shared standing variation in founder populations.
}


In any case, it seems more plausible that adaptation is from standing variation, but picks up different allelic combinations in different high-altitude populations.  That is not at all unlikely, either by chance or because of epistasis which allows the same phenotype to come from different (and incompatible) gene combinations.  Actually, crossing experiments between high-altitude populations would be most informative here. 
\response{
  Great points. See previous note, and our additions to the Discussion.
}


The fixation probability result (attributed to Jagers 1975) is wrong.  Fixation probability is 2s(Ne/N).  For diploid sexualise, Ne=N if the offspring distribution is Poisson with mean 2, in which case the variance of offspring number is 2.  The error may have come from confusion between numbers of genes and numbers of offspring from a sexual pair.
\response{
    Whoops!  We've taken the easy way out, redefining $\xi^2$ to be the ``haploid'' variance in offspring number.
    Hopefully the intent of the reference to Jagers is more clear now 
    (and added better citations, to \cite{fisher1922dominance} and \cite{lambert2006probability};
    probably there is a better reference, but we haven't been able to track it down.
}

\textbf{Reviewer 1: Reviewer's Comments to Author}

I have reviewed the work of Takuno and colleagues on the independent genetic basis of adaptation in maize. The manuscript is well written, the questions that are addressed are well researched and are analyzed in a very thorough and complete fashion. I honestly don't have much to say about it, besides congratulating the authors on producing an excellent and thorough investigation of the demographics and genetic basic of adaptation in maize. The demonstration of the independent origin of highland maize is convincing and the role of adaptive introgression from Zea mays mexicana quite fascinating actually.  
\response{
   We are glad that the reviewer likes our paper.  The comments are very helpful! 
}

Although I read the section on the theoretical evaluation of convergent evolution, followed its logical argument and agreed with it, I did not review it in details.  It is beyond my range of expertise.  I do wonder however how realistic these models are for domesticated species, where census sizes have probably little do to with effective population sizes, and where alleles could potentially be transported several hundreds (or more?) km away in a single generation by humans.  
\response{
    We are glad the theoretical argument was convincing even without the mathematical details.
    The effective population size is properly accounted for here --
    this is, for instance, what the calculation of $\xi^2$ is doing:
    another (less intuitive, in our view) way to phrase this is, as the AE mentions above,
    that the probability of fixation is $2s (N_e/N)$; and $N/N_e \approx \xi^2$.
    We've also added a note to this effect in the text.
}
\response{
While long-distance dispersal is certainly possible, evidence from traditional seed systems in Mexico suggests even today it is rare: when farmers exchange seed (a minority of the time) $\sim 90\%$ of seed lots come from $<10$km away and from a site with altitudinal difference of $<50$m, though farmers in highland locales exchange seeds over a greater range than average \citep{bellon2011}. These numbers represent exchange among modern villages with access to improved means of transportation and a high population density.  While of course it is difficult to estimate ancient patterns of seed migration, these data strongly suggest to us that long-distance transportation (100's of km or greater) must have been exceedingly rare.  Moreover, our previous genetic analysis of an Americas-wide sample of landraces finds no evidence of unusual relatedness among geographically distant samples \citep{vanHeerwaarden_2011_21189301}. We have added reference to \citet{bellon2011} in the text.}


I thought the use of the SNP array + GBS data to be slightly confusing, especially the high heterozygote error rate in the GBS data? However, the authors are aware of this and have treated the data accordingly.  
\response{
We apologize for the lack of clarity.  We have attempted to clarify in the methods that because of the high heterozygous error, we used only the MaizeSNP50 data for running STRUCTURE, but because of ascertainment bias of the MaizeSNP50 data we chose to use only the GBS data for our demographic model. Both data sets were used for identification of outliers.
}
\response{  
It is worth noting, however, that as we showed in Figure S2, despite the very high error rate of calling heterozygote SNPs in GBS, we found that estimated allele frequencies from both GBS and MaizeSNP50 are highly correlated.  Also, we carefully filtered out SNPs that showed departure from Hardy-Weinberg equilibrium (HWE).  We tested different cut-offs of HWE, and obtained  similar results and conclusions. 
 }

Structure analyses:  Using only 3 replicate runs is relatively low, but the results should be robust to this and so I am not too worried.  
\response{
 Although it was not reported in the text, we ran STRUCTURE more than 3 times for preliminary surveys and confirmed that our final result is robust.  We have, however, not modified the text, as the other runs were not on final data but did give identical results.  
 }

The authors do not find much evidence of the same SNPs and genes underlying adaptation in highland south and meso-America. For personal interest, I would be interested to know if by comparing large segments of the genome (megabases scale), patterns would emerge.  This could imply that certain regions of the genome are more prone to be targets of selection  (because of factors such as variations in recombination or mutation rate). This question would be interesting, but not necessary for the current paper.    
\response{
 We agree this is interesting, and have actually tried to test convergent adaptation by sliding window.  However, the density of our SNPs was not high enough to do this (1 SNP/20 kb).  It is very likely that certain genomic regions are more prone to be targets of selection, as low recombination limits the efficacy of selection in some regions.  We suspect careful study of this approach would be better suited for full-genome resequencing, and are currently working on just such an analysis (see \url{https://gcbias.files.wordpress.com/2014/12/beissinger-bapg-2014.pdf}). 
}

This is a minor detail, but you start the abstract by defining convergent evolution, then jump right away to talking about convergent ``adaptation".  Perhaps stick to one or another? 
\response{
We agreed and fixed the abstract.  Thanks!! 
}


\textbf{Reviewer 2: Reviewer's Comments to Author}

Takuno and colleagues use highland and lowland maize populations from sites in Mesoamerica and South America to search for loci involved in adaptation to high altitude.  They identify models to describe demographic history and then use these as the basis for defining putatively adaptive variants based on highland/lowland differentiation.  Somewhat surprisingly, although they find that highly differentiated SNPs tend to come from the standing variation, few of the most differentiated variants are shared among high altitude sites. Instead, the majority of the variants are identified in only one contrasting population set.   
\response{
We appreciate all your valuable suggestions!!
}

The authors put quite a lot of effort into modeling demographic histories and are thus able to use a model-based approach to identify candidate adaptive variants.  This is an improvement over a simple tail-based outlier approach.  However, since the main message of the paper is that convergent rather than parallel evolution drove adaptive response, it is important to ensure that this finding is robust and not simply due to false positives or model mis- specification.  In general, in studies of this sort, it is difficult to assess the relative importance of parallel and convergent adaptive shifts. This is because many of the signals will be due to false positives, and these are reduced in the set of variants that share signatures of divergence between the two altitude contrasts.  
\response{
We agree with the reviewer that this is a potential concern. To assess the effect of false positives, we tried cut-offs of varying stringency.  Using  $F_{ST}$ \textit{P} cutoffs of 0.05, 0.01 and 0.001 we obtained qualitatively similar results, suggesting that the lack of convergent signal is unlikely to be driven by the proportion of false positive hits. We also note that we see enrichment in Mexico for SNPs in regions having been identified as introgressed from ssp. \emph{mexicana} and identified in QTL mapping studies as regions controlling phenotypic differences between high and low elevation teosinte. We have modified text in the discussion to highlight reasons why we believe our results are reflective of genome-wide patterns of adaptation.
}

Information about variant/gene function could help to strengthen the argument -- or to ensure that the argument is the correct one.  There is really no attempt to provide biological context for the results, but this could complement the findings and could potentially help to make them more convincing.  For example, are the loci identified as adaptive enriched for candidate genes or pathways involved in altitude adaptation? How do the set of loci showing parallel versus convergent adaptive signals compare functionally?   
\response{
We agree.  We have now added an analysis of metabolic networks in maize, which are obtained from MaizeCyc database (see text and Table S3). While we see some shared pathways, we do not see more than expected by chance. Unfortunately, gene annotations in the maize reference are somewhate incomplete, such that many of the most interesting loci (red dots on Figure 4) are loci of unknown function, limiting our ability to perform more detailed functional evaluation.
 }

Additional/specific comments:  

To bolster the argument overall, it would be helpful to know whether there is any evidence of convergent adaptation to high altitude at the phenotypic level, for example from transplant experiments or from comparative phenotyping assays in a common garden or other shared environment.  
\response{
 This is a great idea, and actually something we have underway. Our own preliminary common garden experiment suggest a number of phenotypes showing convergent evolution.  We have applied for funding for a larger common garden experiment of crosses between highland and lowland parents, and have a paper underway applying the methods of \citet{berg2013population} to genotyping data to test for convergent evolution of a number of phenotypes from published GWAS.   Unfortunately, our current common garden analyses are not complete enough for publication.  We have, however included citations in the introduction to several sources documenting convergent phenotypes in the two highland populations -- these include macrohairs, stem coloration, tassel architecture, and UV tolerance.  Our own common garden experiment shows convergence for flowering time, plant height, and ear height as well.
 }


There are some very extreme outliers in the differentiation analysis (Figure 4). Do these make biological sense? Also, are relevant pathways/biological processes enriched in the less extreme tail?  Is there greater enrichment among the loci that are shared in the two high altitude populations? I suspect the number of false positive results among those differentiated in both high altitude populations would be lower than those that are differentiated in only one.  
\response{
Unfortunately, the annotation of the maize genome is somewhat incomplete.
There is no annotated function for the few extreme outliers with very small $F_{ST}$ \textit{P}-values.
We have included, however, an analysis of metabolic pathways including genes under selection in Mesoamerica and other genes in S. America.
}

There appears to be strong enrichment of variants with extreme PHS among the variants identified as adaptive. But as far as I could tell, no formal test was conducted. Moreover, I could not find the description of how the authors determined that these had signals from the PHS test. They write that they use the empirical quantile but do not say which quantile is used to define  significance.  
\response{
We clarified our description of the PHS test in Table S2, and the text.  
We used the PHS test \textit{P}-values as an index of haplotype length.
Consider the situation of highland adaptation.
We expected that the PHS test \textit{P}-value of a putative adaptive variant in the highland population ($PHS_1$) is longer than that in the lowland population ($PHS_2$), taking the frequency of the adaptive allele in each population into account.  
Our null expectation is that the probability of $PHS_1>PHS_2$ is 50\% (by chance).  
However, in Table S2, we find that the null expectation is unlikely by a Fisher's exact test and the SNPs involved in highland adaptation tend to have longer haplotype length.  \\
}

Please explain more about the model for fitting genetic distances.  How was it decided that a 10th degree polynomial curve was the best fit for the genetic distances, and how was this model used in the analysis? (Page 5)  
\response{
The PHS test does not need extremely fine-scale recombination rates (nor is such data available). A 10th degree polynomial was simply visually judged to be a good fit to the data; the model was only used to interpolate recombination rates at points along the genome.    
}

It was really interesting that the authors did not find evidence for a domestication bottleneck.  Since this is such a surprising result and one that is in stark contrast to the expectation, it would be helpful to see more about the robustness of this and to have a more complete comparison to previous work.  
\response{
We apologize for the lack of clarity here.  We did not perform any formal testing of the domestication bottleneck, instead, fixing the length of the bottleneck to 1000 years following Wright et al. 2005. While our initial model assumed the length/width ratio of $\sim 2.5$ from Wright et al. 2005, we wanted to test how deviations from this would affect the robustness of our outlier analysis. The salient result is that our finding of little overlap is qualitatively very robust to variation in the details of the domesticaiton bottleneck.  While some of these models gave likelihoods as good or better than the model used in the paper, we felt that detailed estimation of the domestication bottleneck was somewhat outside the scope of the current paper.  We should not thus have made any real claims about the implications of this testing for models of maize domestication; we have removed this section from the paper, leaving only text explaining that the results are robust to deviations from the bottleneck model in Results. 
}

Without some sort of evidence of functional significance, the result that different standing variants increase in frequency with altitude in different populations is not completely satisfying.  Since the variants that increase in frequency are as standing variation in both lowland populations, it seems surprising that the sets of putative adaptive variants are almost completely disjoint.  Without additional evidence for enrichment of the same or similar biological pathways or some other evidence that the outliers are involved in adaptation, this pattern seems most consistent with a finding of different sets of false positives in the two data sets.  
\response{
Even if a variant is adaptive and segregating as standing variation, there is no guarantee it will be selected in both populations.  For a detailed analysis, see \citet{ralph2014role}. We also note that we see a similar lack of convergence regardless of the p-value cutoff we used, suggesting that the observed pattern is unlikely due to the proportion of false positives in our outlier SNPs.  Of course, if all of our results were false positives this would still be the case, but the observed enrichment in regions shown to be introgressed from ssp. \emph{mexicana} and identified as QTL for differences in highland/lowland teosinte subspecies suggests that we have identified at least some biologically meaningful SNPs.  Please also see our response to the above comments regarding biological significance.
}

File S1: Directionality of adaptation contains several typos: \\
shows $<$- show \\
American $<$- America \\
others $<$- other \\
I am not sure what the following sentence means: This pattern could be explained by that the SNP is linked to a read adaptive SNP and recombination breaks down the linkage between them.  
\response{
 We fixed these typos.  Thanks!!   And we apologize for our goof about SNPs showing the pattern described in "Parallel adaptation with recombination".  We found such SNPs in the old version of our data, but after filtering out low-quality SNPs, these were gone.  We have now removed S1, and simplified and clarified our approach here in a new section in the Methods ("Polarizing Adaptation"). 
 }

Figure I: The legend is insufficient to understand what is plotted here. Mainly, in panel A, I cannot find any description of what is represented in each row. Panel B is much more intuitive, but I am still not sure what parallel adaptation with recombination means.  
\response{ 
We have now removed S1, and simplified and clarified our approach here in a new section in the Methods ("Polarizing Adaptation")
}

%\clearpage

\bibliography{MZpara1,MZpara2,plr-hilo}
\bibliographystyle{geneticsT2}

\end{document}










